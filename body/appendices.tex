\section*{附录}
\addcontentsline{toc}{section}{附录}
在本次论文的写的过程,会用到以下算法,现我将简要将这些算法介绍。

\title{附录1  算法类似性}

(1)决策树学习:根据数据的属性采用树状结构建立决策模型。决策树模型常常用来解决分类和回归问题。常见的算法包括 CART (Classification And Regression Tree)、ID3、C4.5、随机森林 (Random Forest) 等。

(2)回归算法:试图采用对误差的衡量来探索变量之间的关系的一类算法。常见的回归算法包括最小二乘法 (Least Square)、逻辑回归 (Logistic Regression)、逐步式回归 (Stepwise Regression) 等。

(3)聚类算法:通常按照中心点或者分层的方式对输入数据进行归并。所有的聚类算法都试图找到数据的内在结构,以便按照最大的共同点将数据进行归类。常见的聚类算法包括 K-Means 算法以及期望最大化算法 (Expectation Maximization) 等。

(4)人工神经网络:模拟生物神经网络,是一类模式匹配算法。通常用于解决分类和回归问题。人工神经网络算法包括感知器神经网络 (Perceptron Neural Network) 、反向传递 (Back Propagation) 和深度学习等。

\newpage
\title{附录2   决策树模型评估}

建立了决策树模型后需要给出该模型的评估值,这样才可以来判断模型的优劣。学习算法模型使用训练集 (training set) 建立模型,使用校验集 (test set) 来评估模型。本文通过评估指标和评估方法来评估决策树模型。 评估指标有分类准确度、召回率、虚警率和精确度等。而这些指标都是基于混淆矩阵 (confusion matrix) 进行计算的。

混淆矩阵是用来评价监督式学习模型的精确性,矩阵的每一列代表一个类的实例预测,而每一行表示一个实际的类的实例。以二类分类问题为例,如下表所示:

\begin{table}[thbp]\footnotesize
	\caption{混淆矩阵}
	\begin{center}
		\begin{tabular}{cc|clc}
			\hline	&预测的类\\
			  实际的类& & 类=1 & 类=0  \\ 
			\hline &类=1 & TP & FN &P\\ 	
			\hline &类=0 & FP & TN&N \\ 
			\hline
		\end{tabular} 
	\end{center}
\end{table}


P (Positive Sample):正例的样本数量。

N(Negative Sample):负例的样本数量。

TP(True Positive):正确预测到的正例的数量。

FP(False Positive):把负例预测成正例的数量。

FN(False Negative):把正例预测成负例的数量。

TN(True Negative):正确预测到的负例的数量。

根据混淆矩阵可以得到评价分类模型的指标有以下几种。

分类准确度,就是正负样本分别被正确分类的概率,计算公式为:

\begin{equation}
Accuracy=\frac{TP+TN}{P+N}
\end{equation}

召回率,就是正样本被识别出的概率,计算公式为:

\begin{equation}
Recall=\frac{TP}{P}
\end{equation}

虚警率,就是负样本被错误分为正样本的概率,计算公式为:

\begin{equation}
FPrate=\frac{FP}{N}
\end{equation}

精确度,就是分类结果为正样本的情况真实性程度,计算公式为:

\begin{equation}
Precision=\frac{TP}{TP+FP}
\end{equation}

评估方法有保留法、随机二次抽样、交叉验证和自助法等。

保留法 (holdout) 是评估分类模型性能的最基本的一种方法。将被标记的原始数据集分成训练集和检验集两份,训练集用于训练分类模型,检验集用于评估分类模型性能。但此方法不适用样本较小的情况,模型可能高度依赖训练集和检验集的构成。

随机二次抽样 (random subsampling) 是指多次重复使用保留方法来改进分类器评估方法。同样此方法也不适用训练集数量不足的情况,而且也可能造成有些数据未被用于训练集。

交叉验证 (cross-validation) 是指把数据分成数量相同的 k 份,每次使用数据进行分类时,选择其中一份作为检验集,剩下的 k-1 份为训练集,重复 k 次,正好使得每一份数据都被用于一次检验集 k-1 次训练集。该方法的优点是尽可能多的数据作为训练集数据,每一次训练集数据和检验集数据都是相互独立的,并且完全覆盖了整个数据集。也存在一个缺点,就是分类模型运行了 K 次,计算开销较大。

自助法 (bootstrap) 是指在其方法中,训练集数据采用的是有放回的抽样,即已经选取为训练集的数据又被放回原来的数据集中,使得该数据有机会能被再一次抽取。用于样本数不多的情况下,效果很好。
\appendix