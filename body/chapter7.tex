\section*{结论}
\addcontentsline{toc}{section}{结论}

根据开发高校毕业设计答辩管理系统的需求,我们首先将功能抽象化,确定了毕业设计答辩工作的流程,以及用户角色划分。采用了JS全栈开发和Meteor开发框架,数据库采用 MongoDB,以及将React作为前端数据渲染引擎。这些都是目前Web开发中最为流行的工具。Node.js作为一个将JavaScript从浏览器带到服务器的项目,现在已经获得肯定并广泛应用于Web开发中。而问答社区Stack Overflow 2016年开发者调查报告中显示,JavaScript 再一次成为了全球最受欢迎的开发技术。

除了可以直接使用来自众多开源工作者的开源项目,JS应用的开发过程也逐渐工业化。包括使用CSS的预处理语言LESS,通过 Babel 工具也能将使用了最新规范ES2015的代码转换为浏览器普遍兼容的ES5代码。将CSS和JS代码批量合并、混淆、压缩,获得了网页更快的加载速度。使用ESLint能将代码出错的可能性降低到最小,并且拥有统一的编码风格,有利于团队协作。同时也介绍了被使用在我们项目中的基于 Git 的工作流。

我们遵循了Meteor和React所提倡的理念:前后端之间只对数据进行交互,对数据的渲染由客户端完成而非服务器端。包括对数据库读写的代码,都在客户端中被执行,节省了发送和接受请求的过程,同时安全性也是有保证的。在实际的开发中,我们感受到了这些特性所带来的便利性,那就是应用逻辑清晰,并且开发速度很快。也正是由于将数据的渲染交给了浏览器,从而大大降低了服务器的负载,在最后的压力测试中表现出了很好的成绩。
